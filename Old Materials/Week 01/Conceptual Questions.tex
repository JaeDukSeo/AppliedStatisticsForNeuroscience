\documentclass{article}
\usepackage{amssymb}
\usepackage{amsmath}
\usepackage{graphicx}
\usepackage{bm}
\usepackage{csquotes}
\newcommand{\tab}{\hspace*{2em}}
\usepackage[normalem]{ulem}

\title{Week 1: Basic Statistics}

\begin{document}

\maketitle

\begin{enumerate}
\item How are descriptive statistics limited?  How can we overcome these limitations?  Consider Anscombe's quartet (see Points of View ``Data Exploration").

\item Briefly discuss each of the following descriptive statistics. How are they useful and what types of data are they most useful for? What are some caveats or limitations to their use?

\begin{enumerate}
	\item  Mean
	\item Median
	\item Standard Deviation
	\item Standard Error of the Mean
	\item Confidence Intervals
	\item Box Plot
\end{enumerate}

\item The most common way to analyze data is to assume that it fits a theoretical distribution (e.g. the normal distribution). This is a big assumption to make. What are some of the caveats?

\item Do think journals should outlaw bar graphs? What about requiring confidence interval error bars instead of s.d. or s.e.m.?

\item Think about your own data. What types of distributions do you typically encounter? How would you know what type of distribution to expect ahead of time? How would you check whether your assumptions about the distribution of your data are correct?

\item Keep thinking about your own data. What are some of your favorite ways to visualize it? What steps do you take to get to know your data before you start testing for significance? Consider any ideas that you want to share with the class in discussion.

\item What are some examples of real data that could be characterized by the following distributions?
\begin{enumerate}
	\item Normal distribution
	\item Poisson distribution
	\item Binomial distribution
	\item Uniform distribution
	\item Irregular distribution
\end{enumerate}

\item Just take a moment to think about how cool the central limit theorem is. It's pretty cool isn't it? Yes, it is. Discuss amongst yourselves.
\end{enumerate}
\end{document}
