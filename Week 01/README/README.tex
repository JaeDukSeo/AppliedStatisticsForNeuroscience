\documentclass[twoside]{article}
\title{Applied Statistics for Neuroscience:\\
	Getting Started}
\usepackage{amssymb}
\usepackage{amsmath}
\usepackage{fancyhdr}
\usepackage{graphicx}
\usepackage{bm}
\pagestyle{fancy}
\pdfpagewidth 8.5in
\pdfpageheight 11in

\begin{document}

\maketitle

You'll need to begin by following the installation instructions in
\texttt{install.pdf}
to set up your computing environment for this course.
We'll be using Python 3 inside of Jupyter notebooks,
along with a variety of standard libraries for working with data.

Once your environment is ready to go
and you've downloaded and unzipped
the file ``Tech Tools Tutorial'' from bCourses,
open up a terminal (OS X/Linux)
or command line (Windows),
and type in

\ \\
\texttt{jupyter notebook}
\ \\

A browser window should open up, showing a file browser.
Navigate to the folder \texttt{Tech Tools Tutorial}
(downloaded from bCourses)
and open up the file
\texttt{Using Jupyter Notebooks.ipynb}.

The homeworks for this course will all be in this format,
and this notebook contains a tutorial on how to use Jupyter notebooks.

The remainder of the notebooks in the folder describe how
to use the other tools we'll be using in the course:

\begin{enumerate}
\item the \textbf{Python} programming language
\item the \textbf{Pandas} data management library
\item the \textbf{Seaborn} plotting package
\item \LaTeX \ typesetting language
\end{enumerate}

listed in order of importance.

\ \\

Reading through the first three tutorials will help you
prepare for the first assignment.

\end{document}
